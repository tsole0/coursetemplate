\documentclass[12pt, a4paper]{article}

% ============================================================================
% SIMPLE MODULAR CLASS NOTES EXAMPLE - PDF/A-3u COMPLIANT
% ============================================================================
% This demonstrates the new modular architecture with PDF/A compliance
% ============================================================================

% PDF/A compliance - MUST be loaded first (includes xcolor and hyperref)
\usepackage[a-3u]{pdfx}

% Font setup for PDF/A compliance
\usepackage{fontspec}
\setmainfont{Charis SIL}  % PDF/A compliant Unicode font

% Essential packages
\usepackage{geometry}
\usepackage[tracking=true]{microtype}
\usepackage{graphicx}

% Load our modular class notes system
\usepackage{course-orgchem}

% Page geometry with wider margins for caution icons
\geometry{
  left=1.2in,
  right=1in,
  top=1.2in,
  bottom=1.2in,
  marginparwidth=1in,
  headheight=15pt
}

% Hyperref is already loaded by pdfx, just configure it
\hypersetup{
  colorlinks=true,
  linkcolor=blue,
  urlcolor=blue,
  citecolor=blue,
  pdfpagemode=UseOutlines,
  bookmarksopen=true
}

% Setup course-specific header
\setuporgchem{Name Here}{Fall 2025}

\begin{document}

\notestitlepage
\tableofcontents
\newpage

\section{Basic Concepts}

\begin{definition}{Functional Group}
A functional group is a specific arrangement of atoms that gives a molecule its characteristic chemical properties.
\end{definition}

\begin{example}[Common Functional Groups]
Key functional groups in organic chemistry:
\begin{itemize}
\item Alcohol: \hydroxylgroup
\item Ketone: \carbonylgroup  
\item Carboxylic acid: \carboxylgroup
\item Amine: \aminogroup
\end{itemize}
\end{example}

\caution[0.5cm] Always identify functional groups first when analyzing a molecule!

\section{Chemical Reactions}

\begin{chemreaction}[Substitution]
Nucleophilic substitution with ethyl bromide:
\begin{center}
\ch{OH- + CH3CH2Br -> CH3CH2OH + Br-}
\end{center}
\end{chemreaction}

\begin{warning}[Laboratory Safety]
Always follow proper safety protocols when working with organic halides.
\end{warning}

\section{Basic Theory}

\begin{theorem}{VSEPR Theory}
Valence Shell Electron Pair Repulsion theory predicts molecular geometry based on minimizing electron pair repulsion around a central atom.
\end{theorem}

\begin{formula}[Bond Angles]
Common bond angles:
\begin{align}
\text{Tetrahedral} &: 109.5° \\
\text{Trigonal planar} &: 120° \\
\text{Linear} &: 180°
\end{align}
\end{formula}

\section{Problem-Solving Strategy}

\begin{note}[Systematic Approach]
When solving organic chemistry problems:
\begin{enumerate}
\item Identify all functional groups
\item Determine the reaction type
\item Consider stereochemistry
\item Check your mechanism
\item Verify your final answer
\end{enumerate}
\end{note}

\marginicon[0.5cm]{\remembersymbol} Practice makes perfect in organic chemistry!

\end{document}
